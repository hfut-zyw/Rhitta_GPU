\documentclass[11pt]{ctexart}

    \usepackage[breakable]{tcolorbox}
    \usepackage{parskip} % Stop auto-indenting (to mimic markdown behaviour)
    

    % Basic figure setup, for now with no caption control since it's done
    % automatically by Pandoc (which extracts ![](path) syntax from Markdown).
    \usepackage{graphicx}
    % Maintain compatibility with old templates. Remove in nbconvert 6.0
    \let\Oldincludegraphics\includegraphics
    % Ensure that by default, figures have no caption (until we provide a
    % proper Figure object with a Caption API and a way to capture that
    % in the conversion process - todo).
    \usepackage{caption}
    \DeclareCaptionFormat{nocaption}{}
    \captionsetup{format=nocaption,aboveskip=0pt,belowskip=0pt}

    \usepackage{float}
    \floatplacement{figure}{H} % forces figures to be placed at the correct location
    \usepackage{xcolor} % Allow colors to be defined
    \usepackage{enumerate} % Needed for markdown enumerations to work
    \usepackage{geometry} % Used to adjust the document margins
    \usepackage{amsmath} % Equations
    \usepackage{amssymb} % Equations
    \usepackage{textcomp} % defines textquotesingle
    % Hack from http://tex.stackexchange.com/a/47451/13684:
    \AtBeginDocument{%
        \def\PYZsq{\textquotesingle}% Upright quotes in Pygmentized code
    }
    \usepackage{upquote} % Upright quotes for verbatim code
    \usepackage{eurosym} % defines \euro

    \usepackage{iftex}
    \ifPDFTeX
        \usepackage[T1]{fontenc}
        \IfFileExists{alphabeta.sty}{
              \usepackage{alphabeta}
          }{
              \usepackage[mathletters]{ucs}
              \usepackage[utf8x]{inputenc}
          }
    \else
        \usepackage{fontspec}
        \usepackage{unicode-math}
    \fi

    \usepackage{fancyvrb} % verbatim replacement that allows latex
    \usepackage{grffile} % extends the file name processing of package graphics
                         % to support a larger range
    \makeatletter % fix for old versions of grffile with XeLaTeX
    \@ifpackagelater{grffile}{2019/11/01}
    {
      % Do nothing on new versions
    }
    {
      \def\Gread@@xetex#1{%
        \IfFileExists{"\Gin@base".bb}%
        {\Gread@eps{\Gin@base.bb}}%
        {\Gread@@xetex@aux#1}%
      }
    }
    \makeatother
    \usepackage[Export]{adjustbox} % Used to constrain images to a maximum size
    \adjustboxset{max size={0.9\linewidth}{0.9\paperheight}}

    % The hyperref package gives us a pdf with properly built
    % internal navigation ('pdf bookmarks' for the table of contents,
    % internal cross-reference links, web links for URLs, etc.)
    \usepackage{hyperref}
    % The default LaTeX title has an obnoxious amount of whitespace. By default,
    % titling removes some of it. It also provides customization options.
    \usepackage{titling}
    \usepackage{longtable} % longtable support required by pandoc >1.10
    \usepackage{booktabs}  % table support for pandoc > 1.12.2
    \usepackage{array}     % table support for pandoc >= 2.11.3
    \usepackage{calc}      % table minipage width calculation for pandoc >= 2.11.1
    \usepackage[inline]{enumitem} % IRkernel/repr support (it uses the enumerate* environment)
    \usepackage[normalem]{ulem} % ulem is needed to support strikethroughs (\sout)
                                % normalem makes italics be italics, not underlines
    \usepackage{mathrsfs}
    

    
    % Colors for the hyperref package
    \definecolor{urlcolor}{rgb}{0,.145,.698}
    \definecolor{linkcolor}{rgb}{.71,0.21,0.01}
    \definecolor{citecolor}{rgb}{.12,.54,.11}

    % ANSI colors
    \definecolor{ansi-black}{HTML}{3E424D}
    \definecolor{ansi-black-intense}{HTML}{282C36}
    \definecolor{ansi-red}{HTML}{E75C58}
    \definecolor{ansi-red-intense}{HTML}{B22B31}
    \definecolor{ansi-green}{HTML}{00A250}
    \definecolor{ansi-green-intense}{HTML}{007427}
    \definecolor{ansi-yellow}{HTML}{DDB62B}
    \definecolor{ansi-yellow-intense}{HTML}{B27D12}
    \definecolor{ansi-blue}{HTML}{208FFB}
    \definecolor{ansi-blue-intense}{HTML}{0065CA}
    \definecolor{ansi-magenta}{HTML}{D160C4}
    \definecolor{ansi-magenta-intense}{HTML}{A03196}
    \definecolor{ansi-cyan}{HTML}{60C6C8}
    \definecolor{ansi-cyan-intense}{HTML}{258F8F}
    \definecolor{ansi-white}{HTML}{C5C1B4}
    \definecolor{ansi-white-intense}{HTML}{A1A6B2}
    \definecolor{ansi-default-inverse-fg}{HTML}{FFFFFF}
    \definecolor{ansi-default-inverse-bg}{HTML}{000000}

    % common color for the border for error outputs.
    \definecolor{outerrorbackground}{HTML}{FFDFDF}

    % commands and environments needed by pandoc snippets
    % extracted from the output of `pandoc -s`
    \providecommand{\tightlist}{%
      \setlength{\itemsep}{0pt}\setlength{\parskip}{0pt}}
    \DefineVerbatimEnvironment{Highlighting}{Verbatim}{commandchars=\\\{\}}
    % Add ',fontsize=\small' for more characters per line
    \newenvironment{Shaded}{}{}
    \newcommand{\KeywordTok}[1]{\textcolor[rgb]{0.00,0.44,0.13}{\textbf{{#1}}}}
    \newcommand{\DataTypeTok}[1]{\textcolor[rgb]{0.56,0.13,0.00}{{#1}}}
    \newcommand{\DecValTok}[1]{\textcolor[rgb]{0.25,0.63,0.44}{{#1}}}
    \newcommand{\BaseNTok}[1]{\textcolor[rgb]{0.25,0.63,0.44}{{#1}}}
    \newcommand{\FloatTok}[1]{\textcolor[rgb]{0.25,0.63,0.44}{{#1}}}
    \newcommand{\CharTok}[1]{\textcolor[rgb]{0.25,0.44,0.63}{{#1}}}
    \newcommand{\StringTok}[1]{\textcolor[rgb]{0.25,0.44,0.63}{{#1}}}
    \newcommand{\CommentTok}[1]{\textcolor[rgb]{0.38,0.63,0.69}{\textit{{#1}}}}
    \newcommand{\OtherTok}[1]{\textcolor[rgb]{0.00,0.44,0.13}{{#1}}}
    \newcommand{\AlertTok}[1]{\textcolor[rgb]{1.00,0.00,0.00}{\textbf{{#1}}}}
    \newcommand{\FunctionTok}[1]{\textcolor[rgb]{0.02,0.16,0.49}{{#1}}}
    \newcommand{\RegionMarkerTok}[1]{{#1}}
    \newcommand{\ErrorTok}[1]{\textcolor[rgb]{1.00,0.00,0.00}{\textbf{{#1}}}}
    \newcommand{\NormalTok}[1]{{#1}}

    % Additional commands for more recent versions of Pandoc
    \newcommand{\ConstantTok}[1]{\textcolor[rgb]{0.53,0.00,0.00}{{#1}}}
    \newcommand{\SpecialCharTok}[1]{\textcolor[rgb]{0.25,0.44,0.63}{{#1}}}
    \newcommand{\VerbatimStringTok}[1]{\textcolor[rgb]{0.25,0.44,0.63}{{#1}}}
    \newcommand{\SpecialStringTok}[1]{\textcolor[rgb]{0.73,0.40,0.53}{{#1}}}
    \newcommand{\ImportTok}[1]{{#1}}
    \newcommand{\DocumentationTok}[1]{\textcolor[rgb]{0.73,0.13,0.13}{\textit{{#1}}}}
    \newcommand{\AnnotationTok}[1]{\textcolor[rgb]{0.38,0.63,0.69}{\textbf{\textit{{#1}}}}}
    \newcommand{\CommentVarTok}[1]{\textcolor[rgb]{0.38,0.63,0.69}{\textbf{\textit{{#1}}}}}
    \newcommand{\VariableTok}[1]{\textcolor[rgb]{0.10,0.09,0.49}{{#1}}}
    \newcommand{\ControlFlowTok}[1]{\textcolor[rgb]{0.00,0.44,0.13}{\textbf{{#1}}}}
    \newcommand{\OperatorTok}[1]{\textcolor[rgb]{0.40,0.40,0.40}{{#1}}}
    \newcommand{\BuiltInTok}[1]{{#1}}
    \newcommand{\ExtensionTok}[1]{{#1}}
    \newcommand{\PreprocessorTok}[1]{\textcolor[rgb]{0.74,0.48,0.00}{{#1}}}
    \newcommand{\AttributeTok}[1]{\textcolor[rgb]{0.49,0.56,0.16}{{#1}}}
    \newcommand{\InformationTok}[1]{\textcolor[rgb]{0.38,0.63,0.69}{\textbf{\textit{{#1}}}}}
    \newcommand{\WarningTok}[1]{\textcolor[rgb]{0.38,0.63,0.69}{\textbf{\textit{{#1}}}}}


    % Define a nice break command that doesn't care if a line doesn't already
    % exist.
    \def\br{\hspace*{\fill} \\* }
    % Math Jax compatibility definitions
    \def\gt{>}
    \def\lt{<}
    \let\Oldtex\TeX
    \let\Oldlatex\LaTeX
    \renewcommand{\TeX}{\textrm{\Oldtex}}
    \renewcommand{\LaTeX}{\textrm{\Oldlatex}}
    % Document parameters
    % Document title
    \title{案例5:使用BiLSTM对IMDB做分类}
    
    
    
    
    
% Pygments definitions
\makeatletter
\def\PY@reset{\let\PY@it=\relax \let\PY@bf=\relax%
    \let\PY@ul=\relax \let\PY@tc=\relax%
    \let\PY@bc=\relax \let\PY@ff=\relax}
\def\PY@tok#1{\csname PY@tok@#1\endcsname}
\def\PY@toks#1+{\ifx\relax#1\empty\else%
    \PY@tok{#1}\expandafter\PY@toks\fi}
\def\PY@do#1{\PY@bc{\PY@tc{\PY@ul{%
    \PY@it{\PY@bf{\PY@ff{#1}}}}}}}
\def\PY#1#2{\PY@reset\PY@toks#1+\relax+\PY@do{#2}}

\@namedef{PY@tok@w}{\def\PY@tc##1{\textcolor[rgb]{0.73,0.73,0.73}{##1}}}
\@namedef{PY@tok@c}{\let\PY@it=\textit\def\PY@tc##1{\textcolor[rgb]{0.24,0.48,0.48}{##1}}}
\@namedef{PY@tok@cp}{\def\PY@tc##1{\textcolor[rgb]{0.61,0.40,0.00}{##1}}}
\@namedef{PY@tok@k}{\let\PY@bf=\textbf\def\PY@tc##1{\textcolor[rgb]{0.00,0.50,0.00}{##1}}}
\@namedef{PY@tok@kp}{\def\PY@tc##1{\textcolor[rgb]{0.00,0.50,0.00}{##1}}}
\@namedef{PY@tok@kt}{\def\PY@tc##1{\textcolor[rgb]{0.69,0.00,0.25}{##1}}}
\@namedef{PY@tok@o}{\def\PY@tc##1{\textcolor[rgb]{0.40,0.40,0.40}{##1}}}
\@namedef{PY@tok@ow}{\let\PY@bf=\textbf\def\PY@tc##1{\textcolor[rgb]{0.67,0.13,1.00}{##1}}}
\@namedef{PY@tok@nb}{\def\PY@tc##1{\textcolor[rgb]{0.00,0.50,0.00}{##1}}}
\@namedef{PY@tok@nf}{\def\PY@tc##1{\textcolor[rgb]{0.00,0.00,1.00}{##1}}}
\@namedef{PY@tok@nc}{\let\PY@bf=\textbf\def\PY@tc##1{\textcolor[rgb]{0.00,0.00,1.00}{##1}}}
\@namedef{PY@tok@nn}{\let\PY@bf=\textbf\def\PY@tc##1{\textcolor[rgb]{0.00,0.00,1.00}{##1}}}
\@namedef{PY@tok@ne}{\let\PY@bf=\textbf\def\PY@tc##1{\textcolor[rgb]{0.80,0.25,0.22}{##1}}}
\@namedef{PY@tok@nv}{\def\PY@tc##1{\textcolor[rgb]{0.10,0.09,0.49}{##1}}}
\@namedef{PY@tok@no}{\def\PY@tc##1{\textcolor[rgb]{0.53,0.00,0.00}{##1}}}
\@namedef{PY@tok@nl}{\def\PY@tc##1{\textcolor[rgb]{0.46,0.46,0.00}{##1}}}
\@namedef{PY@tok@ni}{\let\PY@bf=\textbf\def\PY@tc##1{\textcolor[rgb]{0.44,0.44,0.44}{##1}}}
\@namedef{PY@tok@na}{\def\PY@tc##1{\textcolor[rgb]{0.41,0.47,0.13}{##1}}}
\@namedef{PY@tok@nt}{\let\PY@bf=\textbf\def\PY@tc##1{\textcolor[rgb]{0.00,0.50,0.00}{##1}}}
\@namedef{PY@tok@nd}{\def\PY@tc##1{\textcolor[rgb]{0.67,0.13,1.00}{##1}}}
\@namedef{PY@tok@s}{\def\PY@tc##1{\textcolor[rgb]{0.73,0.13,0.13}{##1}}}
\@namedef{PY@tok@sd}{\let\PY@it=\textit\def\PY@tc##1{\textcolor[rgb]{0.73,0.13,0.13}{##1}}}
\@namedef{PY@tok@si}{\let\PY@bf=\textbf\def\PY@tc##1{\textcolor[rgb]{0.64,0.35,0.47}{##1}}}
\@namedef{PY@tok@se}{\let\PY@bf=\textbf\def\PY@tc##1{\textcolor[rgb]{0.67,0.36,0.12}{##1}}}
\@namedef{PY@tok@sr}{\def\PY@tc##1{\textcolor[rgb]{0.64,0.35,0.47}{##1}}}
\@namedef{PY@tok@ss}{\def\PY@tc##1{\textcolor[rgb]{0.10,0.09,0.49}{##1}}}
\@namedef{PY@tok@sx}{\def\PY@tc##1{\textcolor[rgb]{0.00,0.50,0.00}{##1}}}
\@namedef{PY@tok@m}{\def\PY@tc##1{\textcolor[rgb]{0.40,0.40,0.40}{##1}}}
\@namedef{PY@tok@gh}{\let\PY@bf=\textbf\def\PY@tc##1{\textcolor[rgb]{0.00,0.00,0.50}{##1}}}
\@namedef{PY@tok@gu}{\let\PY@bf=\textbf\def\PY@tc##1{\textcolor[rgb]{0.50,0.00,0.50}{##1}}}
\@namedef{PY@tok@gd}{\def\PY@tc##1{\textcolor[rgb]{0.63,0.00,0.00}{##1}}}
\@namedef{PY@tok@gi}{\def\PY@tc##1{\textcolor[rgb]{0.00,0.52,0.00}{##1}}}
\@namedef{PY@tok@gr}{\def\PY@tc##1{\textcolor[rgb]{0.89,0.00,0.00}{##1}}}
\@namedef{PY@tok@ge}{\let\PY@it=\textit}
\@namedef{PY@tok@gs}{\let\PY@bf=\textbf}
\@namedef{PY@tok@gp}{\let\PY@bf=\textbf\def\PY@tc##1{\textcolor[rgb]{0.00,0.00,0.50}{##1}}}
\@namedef{PY@tok@go}{\def\PY@tc##1{\textcolor[rgb]{0.44,0.44,0.44}{##1}}}
\@namedef{PY@tok@gt}{\def\PY@tc##1{\textcolor[rgb]{0.00,0.27,0.87}{##1}}}
\@namedef{PY@tok@err}{\def\PY@bc##1{{\setlength{\fboxsep}{\string -\fboxrule}\fcolorbox[rgb]{1.00,0.00,0.00}{1,1,1}{\strut ##1}}}}
\@namedef{PY@tok@kc}{\let\PY@bf=\textbf\def\PY@tc##1{\textcolor[rgb]{0.00,0.50,0.00}{##1}}}
\@namedef{PY@tok@kd}{\let\PY@bf=\textbf\def\PY@tc##1{\textcolor[rgb]{0.00,0.50,0.00}{##1}}}
\@namedef{PY@tok@kn}{\let\PY@bf=\textbf\def\PY@tc##1{\textcolor[rgb]{0.00,0.50,0.00}{##1}}}
\@namedef{PY@tok@kr}{\let\PY@bf=\textbf\def\PY@tc##1{\textcolor[rgb]{0.00,0.50,0.00}{##1}}}
\@namedef{PY@tok@bp}{\def\PY@tc##1{\textcolor[rgb]{0.00,0.50,0.00}{##1}}}
\@namedef{PY@tok@fm}{\def\PY@tc##1{\textcolor[rgb]{0.00,0.00,1.00}{##1}}}
\@namedef{PY@tok@vc}{\def\PY@tc##1{\textcolor[rgb]{0.10,0.09,0.49}{##1}}}
\@namedef{PY@tok@vg}{\def\PY@tc##1{\textcolor[rgb]{0.10,0.09,0.49}{##1}}}
\@namedef{PY@tok@vi}{\def\PY@tc##1{\textcolor[rgb]{0.10,0.09,0.49}{##1}}}
\@namedef{PY@tok@vm}{\def\PY@tc##1{\textcolor[rgb]{0.10,0.09,0.49}{##1}}}
\@namedef{PY@tok@sa}{\def\PY@tc##1{\textcolor[rgb]{0.73,0.13,0.13}{##1}}}
\@namedef{PY@tok@sb}{\def\PY@tc##1{\textcolor[rgb]{0.73,0.13,0.13}{##1}}}
\@namedef{PY@tok@sc}{\def\PY@tc##1{\textcolor[rgb]{0.73,0.13,0.13}{##1}}}
\@namedef{PY@tok@dl}{\def\PY@tc##1{\textcolor[rgb]{0.73,0.13,0.13}{##1}}}
\@namedef{PY@tok@s2}{\def\PY@tc##1{\textcolor[rgb]{0.73,0.13,0.13}{##1}}}
\@namedef{PY@tok@sh}{\def\PY@tc##1{\textcolor[rgb]{0.73,0.13,0.13}{##1}}}
\@namedef{PY@tok@s1}{\def\PY@tc##1{\textcolor[rgb]{0.73,0.13,0.13}{##1}}}
\@namedef{PY@tok@mb}{\def\PY@tc##1{\textcolor[rgb]{0.40,0.40,0.40}{##1}}}
\@namedef{PY@tok@mf}{\def\PY@tc##1{\textcolor[rgb]{0.40,0.40,0.40}{##1}}}
\@namedef{PY@tok@mh}{\def\PY@tc##1{\textcolor[rgb]{0.40,0.40,0.40}{##1}}}
\@namedef{PY@tok@mi}{\def\PY@tc##1{\textcolor[rgb]{0.40,0.40,0.40}{##1}}}
\@namedef{PY@tok@il}{\def\PY@tc##1{\textcolor[rgb]{0.40,0.40,0.40}{##1}}}
\@namedef{PY@tok@mo}{\def\PY@tc##1{\textcolor[rgb]{0.40,0.40,0.40}{##1}}}
\@namedef{PY@tok@ch}{\let\PY@it=\textit\def\PY@tc##1{\textcolor[rgb]{0.24,0.48,0.48}{##1}}}
\@namedef{PY@tok@cm}{\let\PY@it=\textit\def\PY@tc##1{\textcolor[rgb]{0.24,0.48,0.48}{##1}}}
\@namedef{PY@tok@cpf}{\let\PY@it=\textit\def\PY@tc##1{\textcolor[rgb]{0.24,0.48,0.48}{##1}}}
\@namedef{PY@tok@c1}{\let\PY@it=\textit\def\PY@tc##1{\textcolor[rgb]{0.24,0.48,0.48}{##1}}}
\@namedef{PY@tok@cs}{\let\PY@it=\textit\def\PY@tc##1{\textcolor[rgb]{0.24,0.48,0.48}{##1}}}

\def\PYZbs{\char`\\}
\def\PYZus{\char`\_}
\def\PYZob{\char`\{}
\def\PYZcb{\char`\}}
\def\PYZca{\char`\^}
\def\PYZam{\char`\&}
\def\PYZlt{\char`\<}
\def\PYZgt{\char`\>}
\def\PYZsh{\char`\#}
\def\PYZpc{\char`\%}
\def\PYZdl{\char`\$}
\def\PYZhy{\char`\-}
\def\PYZsq{\char`\'}
\def\PYZdq{\char`\"}
\def\PYZti{\char`\~}
% for compatibility with earlier versions
\def\PYZat{@}
\def\PYZlb{[}
\def\PYZrb{]}
\makeatother


    % For linebreaks inside Verbatim environment from package fancyvrb.
    \makeatletter
        \newbox\Wrappedcontinuationbox
        \newbox\Wrappedvisiblespacebox
        \newcommand*\Wrappedvisiblespace {\textcolor{red}{\textvisiblespace}}
        \newcommand*\Wrappedcontinuationsymbol {\textcolor{red}{\llap{\tiny$\m@th\hookrightarrow$}}}
        \newcommand*\Wrappedcontinuationindent {3ex }
        \newcommand*\Wrappedafterbreak {\kern\Wrappedcontinuationindent\copy\Wrappedcontinuationbox}
        % Take advantage of the already applied Pygments mark-up to insert
        % potential linebreaks for TeX processing.
        %        {, <, #, %, $, ' and ": go to next line.
        %        _, }, ^, &, >, - and ~: stay at end of broken line.
        % Use of \textquotesingle for straight quote.
        \newcommand*\Wrappedbreaksatspecials {%
            \def\PYGZus{\discretionary{\char`\_}{\Wrappedafterbreak}{\char`\_}}%
            \def\PYGZob{\discretionary{}{\Wrappedafterbreak\char`\{}{\char`\{}}%
            \def\PYGZcb{\discretionary{\char`\}}{\Wrappedafterbreak}{\char`\}}}%
            \def\PYGZca{\discretionary{\char`\^}{\Wrappedafterbreak}{\char`\^}}%
            \def\PYGZam{\discretionary{\char`\&}{\Wrappedafterbreak}{\char`\&}}%
            \def\PYGZlt{\discretionary{}{\Wrappedafterbreak\char`\<}{\char`\<}}%
            \def\PYGZgt{\discretionary{\char`\>}{\Wrappedafterbreak}{\char`\>}}%
            \def\PYGZsh{\discretionary{}{\Wrappedafterbreak\char`\#}{\char`\#}}%
            \def\PYGZpc{\discretionary{}{\Wrappedafterbreak\char`\%}{\char`\%}}%
            \def\PYGZdl{\discretionary{}{\Wrappedafterbreak\char`\$}{\char`\$}}%
            \def\PYGZhy{\discretionary{\char`\-}{\Wrappedafterbreak}{\char`\-}}%
            \def\PYGZsq{\discretionary{}{\Wrappedafterbreak\textquotesingle}{\textquotesingle}}%
            \def\PYGZdq{\discretionary{}{\Wrappedafterbreak\char`\"}{\char`\"}}%
            \def\PYGZti{\discretionary{\char`\~}{\Wrappedafterbreak}{\char`\~}}%
        }
        % Some characters . , ; ? ! / are not pygmentized.
        % This macro makes them "active" and they will insert potential linebreaks
        \newcommand*\Wrappedbreaksatpunct {%
            \lccode`\~`\.\lowercase{\def~}{\discretionary{\hbox{\char`\.}}{\Wrappedafterbreak}{\hbox{\char`\.}}}%
            \lccode`\~`\,\lowercase{\def~}{\discretionary{\hbox{\char`\,}}{\Wrappedafterbreak}{\hbox{\char`\,}}}%
            \lccode`\~`\;\lowercase{\def~}{\discretionary{\hbox{\char`\;}}{\Wrappedafterbreak}{\hbox{\char`\;}}}%
            \lccode`\~`\:\lowercase{\def~}{\discretionary{\hbox{\char`\:}}{\Wrappedafterbreak}{\hbox{\char`\:}}}%
            \lccode`\~`\?\lowercase{\def~}{\discretionary{\hbox{\char`\?}}{\Wrappedafterbreak}{\hbox{\char`\?}}}%
            \lccode`\~`\!\lowercase{\def~}{\discretionary{\hbox{\char`\!}}{\Wrappedafterbreak}{\hbox{\char`\!}}}%
            \lccode`\~`\/\lowercase{\def~}{\discretionary{\hbox{\char`\/}}{\Wrappedafterbreak}{\hbox{\char`\/}}}%
            \catcode`\.\active
            \catcode`\,\active
            \catcode`\;\active
            \catcode`\:\active
            \catcode`\?\active
            \catcode`\!\active
            \catcode`\/\active
            \lccode`\~`\~
        }
    \makeatother

    \let\OriginalVerbatim=\Verbatim
    \makeatletter
    \renewcommand{\Verbatim}[1][1]{%
        %\parskip\z@skip
        \sbox\Wrappedcontinuationbox {\Wrappedcontinuationsymbol}%
        \sbox\Wrappedvisiblespacebox {\FV@SetupFont\Wrappedvisiblespace}%
        \def\FancyVerbFormatLine ##1{\hsize\linewidth
            \vtop{\raggedright\hyphenpenalty\z@\exhyphenpenalty\z@
                \doublehyphendemerits\z@\finalhyphendemerits\z@
                \strut ##1\strut}%
        }%
        % If the linebreak is at a space, the latter will be displayed as visible
        % space at end of first line, and a continuation symbol starts next line.
        % Stretch/shrink are however usually zero for typewriter font.
        \def\FV@Space {%
            \nobreak\hskip\z@ plus\fontdimen3\font minus\fontdimen4\font
            \discretionary{\copy\Wrappedvisiblespacebox}{\Wrappedafterbreak}
            {\kern\fontdimen2\font}%
        }%

        % Allow breaks at special characters using \PYG... macros.
        \Wrappedbreaksatspecials
        % Breaks at punctuation characters . , ; ? ! and / need catcode=\active
        \OriginalVerbatim[#1,codes*=\Wrappedbreaksatpunct]%
    }
    \makeatother

    % Exact colors from NB
    \definecolor{incolor}{HTML}{303F9F}
    \definecolor{outcolor}{HTML}{D84315}
    \definecolor{cellborder}{HTML}{CFCFCF}
    \definecolor{cellbackground}{HTML}{F7F7F7}

    % prompt
    \makeatletter
    \newcommand{\boxspacing}{\kern\kvtcb@left@rule\kern\kvtcb@boxsep}
    \makeatother
    \newcommand{\prompt}[4]{
        {\ttfamily\llap{{\color{#2}[#3]:\hspace{3pt}#4}}\vspace{-\baselineskip}}
    }
    

    
    % Prevent overflowing lines due to hard-to-break entities
    \sloppy
    % Setup hyperref package
    \hypersetup{
      breaklinks=true,  % so long urls are correctly broken across lines
      colorlinks=true,
      urlcolor=urlcolor,
      linkcolor=linkcolor,
      citecolor=citecolor,
      }
    % Slightly bigger margins than the latex defaults
    
    \geometry{verbose,tmargin=1in,bmargin=1in,lmargin=1in,rmargin=1in}
    
    

\begin{document}
    
    \maketitle
    
    

    
    \hypertarget{ux6848ux4f8b5ux4f7fux7528bilstmux5bf9imdbux505aux5206ux7c7b}{%
\label{ux6848ux4f8b5ux4f7fux7528bilstmux5bf9imdbux505aux5206ux7c7b}}

    由于本框架计算太慢,句子长度不能设置太大,实际上大多数句子长度\\
都超过200了,仅仅依靠前面几个词根本无法预测,这里仅仅是展示模型是否跑的通\\
后面的解释可能会按256个词来解释

本案例为了能跑通,更改了以下几个参数:\\
1.max\_seq\_len=256 --\textgreater{} 8\\
2.vocab\_lens=len(loader.dict) --\textgreater{} 20
(这样单词id会溢出词典,注意)\\
3.训练的样本数量:N= len(seqs) --\textgreater{} 12

    \begin{tcolorbox}[breakable, size=fbox, boxrule=1pt, pad at break*=1mm,colback=cellbackground, colframe=cellborder]
\prompt{In}{incolor}{1}{\boxspacing}
\begin{Verbatim}[commandchars=\\\{\}]
\PY{k+kn}{import} \PY{n+nn}{sys}
\PY{n}{sys}\PY{o}{.}\PY{n}{path}\PY{o}{.}\PY{n}{append}\PY{p}{(}\PY{l+s+sa}{r}\PY{l+s+s2}{\PYZdq{}}\PY{l+s+s2}{D:}\PY{l+s+s2}{\PYZbs{}}\PY{l+s+s2}{Rhitta\PYZus{}GPU}\PY{l+s+s2}{\PYZdq{}}\PY{p}{)}
\PY{k+kn}{import} \PY{n+nn}{numpy} \PY{k}{as} \PY{n+nn}{np}
\PY{k+kn}{import} \PY{n+nn}{cupy} \PY{k}{as} \PY{n+nn}{cp}
\PY{k+kn}{import} \PY{n+nn}{rhitta}\PY{n+nn}{.}\PY{n+nn}{nn} \PY{k}{as} \PY{n+nn}{nn}
\end{Verbatim}
\end{tcolorbox}

    \hypertarget{ux7b2cux4e00ux6b65ux8f7dux5165ux6570ux636eux96c6}{%
\subsubsection*{第一步:载入数据集}\label{ux7b2cux4e00ux6b65ux8f7dux5165ux6570ux636eux96c6}}

会获得两个列表,分别是句子和标签\\
IMDBLoader()接口接收一个指定句子长度的参数max\_seq\_len

    \begin{tcolorbox}[breakable, size=fbox, boxrule=1pt, pad at break*=1mm,colback=cellbackground, colframe=cellborder]
\prompt{In}{incolor}{2}{\boxspacing}
\begin{Verbatim}[commandchars=\\\{\}]
\PY{n}{max\PYZus{}seq\PYZus{}len} \PY{o}{=} \PY{l+m+mi}{8} \PY{c+c1}{\PYZsh{} 设置每个句子的长度}
\PY{n}{loader}\PY{o}{=}\PY{n}{nn}\PY{o}{.}\PY{n}{IMDBLoader}\PY{p}{(}\PY{n}{max\PYZus{}seq\PYZus{}len}\PY{o}{=}\PY{n}{max\PYZus{}seq\PYZus{}len}\PY{p}{)}
\PY{n}{seqs}\PY{p}{,}\PY{n}{labels}\PY{o}{=}\PY{n}{loader}\PY{o}{.}\PY{n}{load}\PY{p}{(}\PY{l+s+sa}{r}\PY{l+s+s2}{\PYZdq{}}\PY{l+s+s2}{D:}\PY{l+s+s2}{\PYZbs{}}\PY{l+s+s2}{Rhitta\PYZus{}GPU}\PY{l+s+s2}{\PYZbs{}}\PY{l+s+s2}{data}\PY{l+s+s2}{\PYZbs{}}\PY{l+s+s2}{dataset}\PY{l+s+s2}{\PYZdq{}}\PY{p}{)}
\end{Verbatim}
\end{tcolorbox}

    \begin{tcolorbox}[breakable, size=fbox, boxrule=1pt, pad at break*=1mm,colback=cellbackground, colframe=cellborder]
\prompt{In}{incolor}{3}{\boxspacing}
\begin{Verbatim}[commandchars=\\\{\}]
\PY{c+c1}{\PYZsh{}\PYZsh{} test }
\PY{k}{for} \PY{n}{i} \PY{o+ow}{in} \PY{n+nb}{range}\PY{p}{(}\PY{l+m+mi}{10}\PY{p}{)}\PY{p}{:}
    \PY{n+nb}{print}\PY{p}{(}\PY{n}{seqs}\PY{p}{[}\PY{n}{i}\PY{p}{]}\PY{p}{,}\PY{n}{labels}\PY{p}{[}\PY{n}{i}\PY{p}{]}\PY{p}{)}
\end{Verbatim}
\end{tcolorbox}

    \begin{Verbatim}[commandchars=\\\{\}]
[552, 35, 14, 554, 171, 188, 34, 2] 1
[16, 5344, 946, 5, 2, 10059, 33, 3898] 1
[16, 21, 529, 19, 4, 3179, 4752, 5200] 0
[41, 218, 30, 5201, 12, 162, 37, 5812] 0
[25376, 4556, 3, 44405, 12118, 13, 2284, 37792] 1
[10, 6192, 123, 176, 83, 105, 14, 7] 1
[49, 18, 97, 29, 77, 54, 50, 22] 0
[10, 301, 14, 398, 34, 4, 464, 13] 0
[112, 464, 5, 1966, 1249, 14, 21, 20] 0
[8, 8, 49, 18, 7, 407, 5, 2124] 1
    \end{Verbatim}

    \hypertarget{ux7b2cux4e8cux6b65ux6784ux9020ux6a21ux578b}{%
\subsubsection*{第二步:构造模型}\label{ux7b2cux4e8cux6b65ux6784ux9020ux6a21ux578b}}

先把句子的256个词,也就是256个数字丢进embedding层,变成256个向量,\\
再把256个向量丢进BiLSTM中,获得一个输出,最后送入分类头\\
注意:BiLSTM最后的汇聚层需要忽略掉pad过来的向量,\\
由于实现起来有些繁琐,这里就不忽略了,影响不大\\
不忽略相当于后面的神经元用于传递之前的信息,没有新信息加入

    \begin{tcolorbox}[breakable, size=fbox, boxrule=1pt, pad at break*=1mm,colback=cellbackground, colframe=cellborder]
\prompt{In}{incolor}{4}{\boxspacing}
\begin{Verbatim}[commandchars=\\\{\}]
\PY{k}{class} \PY{n+nc}{zyw}\PY{p}{(}\PY{n}{nn}\PY{o}{.}\PY{n}{Module}\PY{p}{)}\PY{p}{:}
    \PY{k}{def} \PY{n+nf+fm}{\PYZus{}\PYZus{}init\PYZus{}\PYZus{}}\PY{p}{(}\PY{n+nb+bp}{self}\PY{p}{)}\PY{p}{:}
        \PY{n+nb}{super}\PY{p}{(}\PY{n}{zyw}\PY{p}{,}\PY{n+nb+bp}{self}\PY{p}{)}\PY{o}{.}\PY{n+nf+fm}{\PYZus{}\PYZus{}init\PYZus{}\PYZus{}}\PY{p}{(}\PY{p}{)}        
        \PY{n+nb+bp}{self}\PY{o}{.}\PY{n}{bilstm}\PY{o}{=}\PY{n}{nn}\PY{o}{.}\PY{n}{BiLSTM}\PY{p}{(}\PY{n}{input\PYZus{}size}\PY{o}{=}\PY{l+m+mi}{6}\PY{p}{,}\PY{n}{hidden\PYZus{}size}\PY{o}{=}\PY{l+m+mi}{4}\PY{p}{,}\PY{n}{time\PYZus{}dimension}\PY{o}{=}\PY{n}{max\PYZus{}seq\PYZus{}len}\PY{p}{,}\PY{n}{mode}\PY{o}{=}\PY{l+m+mi}{1}\PY{p}{)}
        \PY{n+nb+bp}{self}\PY{o}{.}\PY{n}{linear}\PY{o}{=}\PY{n}{nn}\PY{o}{.}\PY{n}{Linear}\PY{p}{(}\PY{l+m+mi}{8}\PY{p}{,}\PY{l+m+mi}{1}\PY{p}{,}\PY{n}{activation} \PY{o}{=} \PY{l+s+s2}{\PYZdq{}}\PY{l+s+s2}{Logistic}\PY{l+s+s2}{\PYZdq{}}\PY{p}{)} \PY{c+c1}{\PYZsh{} 注意BiLSTM把隐藏层拼接了,向量维度变成2倍了}
    \PY{k}{def} \PY{n+nf+fm}{\PYZus{}\PYZus{}call\PYZus{}\PYZus{}}\PY{p}{(}\PY{n+nb+bp}{self}\PY{p}{,}\PY{n}{seq\PYZus{}embeddings}\PY{p}{,}\PY{n}{h\PYZus{}0}\PY{p}{,}\PY{n}{c\PYZus{}0}\PY{p}{,}\PY{n}{h\PYZus{}1}\PY{p}{,}\PY{n}{c\PYZus{}1}\PY{p}{)}\PY{p}{:}
        \PY{n}{x}\PY{o}{=}\PY{n+nb+bp}{self}\PY{o}{.}\PY{n}{bilstm}\PY{p}{(}\PY{n}{seq\PYZus{}embeddings}\PY{p}{,}\PY{n}{h\PYZus{}0}\PY{p}{,}\PY{n}{c\PYZus{}0}\PY{p}{,}\PY{n}{h\PYZus{}1}\PY{p}{,}\PY{n}{c\PYZus{}1}\PY{p}{)}
        \PY{n}{x}\PY{o}{=}\PY{n+nb+bp}{self}\PY{o}{.}\PY{n}{linear}\PY{p}{(}\PY{n}{x}\PY{p}{)}
        \PY{k}{return} \PY{n}{x}
\PY{n}{vocab\PYZus{}lens}\PY{o}{=}\PY{n+nb}{len}\PY{p}{(}\PY{n}{loader}\PY{o}{.}\PY{n}{dict}\PY{p}{)}
\PY{n}{vocab\PYZus{}lens}\PY{o}{=}\PY{l+m+mi}{20} \PY{c+c1}{\PYZsh{} 字典太大训练不动,但是取词的时候,词的id很容易超过这个数}
\PY{n}{embedding}\PY{o}{=}\PY{n}{nn}\PY{o}{.}\PY{n}{Embedding}\PY{p}{(}\PY{n}{numembeddings}\PY{o}{=}\PY{n}{vocab\PYZus{}lens}\PY{p}{,} \PY{n}{embeddingdim}\PY{o}{=}\PY{l+m+mi}{6}\PY{p}{,} \PY{n}{paddingidx}\PY{o}{=}\PY{l+m+mi}{0}\PY{p}{)}
\PY{n}{model} \PY{o}{=} \PY{n}{zyw}\PY{p}{(}\PY{p}{)}
\end{Verbatim}
\end{tcolorbox}

    \hypertarget{ux7b2cux4e09ux6b65ux6784ux9020ux8ba1ux7b97ux56fe}{%
\paragraph{第三步:构造计算图\newline}\label{ux7b2cux4e09ux6b65ux6784ux9020ux8ba1ux7b97ux56fe}}


坑节点包括:

句子列表:必须是一个固定不动的对象,后面需要往里面填写每个句子的数字\\
embedding一旦实例化,就不能变动,只能改输入对象的内部数值

初始隐藏状态节点:由于是双向LSTM,需要4个,形状(1,4)\\
标签节点:由于是二分类,形状为(1,1)

    \begin{tcolorbox}[breakable, size=fbox, boxrule=1pt, pad at break*=1mm,colback=cellbackground, colframe=cellborder]
\prompt{In}{incolor}{5}{\boxspacing}
\begin{Verbatim}[commandchars=\\\{\}]
\PY{c+c1}{\PYZsh{} 构造坑位,注意,叶子节点不是输入的列表,而是编码器里面的词典,已经自动创建好了}
\PY{c+c1}{\PYZsh{} 当词典更新set\PYZus{}value时,所有下游节点全部reset\PYZus{}value}
\PY{n}{seq} \PY{o}{=} \PY{p}{[}\PY{n}{i} \PY{k}{for} \PY{n}{i} \PY{o+ow}{in} \PY{n+nb}{range}\PY{p}{(}\PY{n}{max\PYZus{}seq\PYZus{}len}\PY{p}{)}\PY{p}{]}
\PY{n}{h\PYZus{}0}\PY{p}{,}\PY{n}{c\PYZus{}0}\PY{p}{,}\PY{n}{h\PYZus{}1}\PY{p}{,}\PY{n}{c\PYZus{}1}\PY{o}{=}\PY{n}{nn}\PY{o}{.}\PY{n}{to\PYZus{}tensor}\PY{p}{(}\PY{p}{(}\PY{l+m+mi}{1}\PY{p}{,}\PY{l+m+mi}{4}\PY{p}{)}\PY{p}{)}\PY{p}{,}\PY{n}{nn}\PY{o}{.}\PY{n}{to\PYZus{}tensor}\PY{p}{(}\PY{p}{(}\PY{l+m+mi}{1}\PY{p}{,}\PY{l+m+mi}{4}\PY{p}{)}\PY{p}{)}\PY{p}{,}\PY{n}{nn}\PY{o}{.}\PY{n}{to\PYZus{}tensor}\PY{p}{(}\PY{p}{(}\PY{l+m+mi}{1}\PY{p}{,}\PY{l+m+mi}{4}\PY{p}{)}\PY{p}{)}\PY{p}{,}\PY{n}{nn}\PY{o}{.}\PY{n}{to\PYZus{}tensor}\PY{p}{(}\PY{p}{(}\PY{l+m+mi}{1}\PY{p}{,}\PY{l+m+mi}{4}\PY{p}{)}\PY{p}{)}
\PY{n}{label} \PY{o}{=} \PY{n}{nn}\PY{o}{.}\PY{n}{to\PYZus{}tensor}\PY{p}{(}\PY{p}{(}\PY{l+m+mi}{1}\PY{p}{,}\PY{l+m+mi}{1}\PY{p}{)}\PY{p}{)}

\PY{c+c1}{\PYZsh{} 构造计算图}
\PY{n}{seq\PYZus{}embedding} \PY{o}{=} \PY{n}{embedding}\PY{p}{(}\PY{n}{seq}\PY{p}{)}
\PY{n}{output} \PY{o}{=} \PY{n}{model}\PY{p}{(}\PY{n}{seq\PYZus{}embedding}\PY{p}{,}\PY{n}{h\PYZus{}0}\PY{p}{,}\PY{n}{c\PYZus{}0}\PY{p}{,}\PY{n}{h\PYZus{}1}\PY{p}{,}\PY{n}{c\PYZus{}1}\PY{p}{)}
\PY{n}{loss} \PY{o}{=} \PY{n}{nn}\PY{o}{.}\PY{n}{BinaryClassLoss}\PY{p}{(}\PY{n}{output}\PY{p}{,}\PY{n}{label}\PY{p}{)}
\end{Verbatim}
\end{tcolorbox}

    \hypertarget{ux7b2cux56dbux6b65ux521dux59cbux5316ux4f18ux5316ux5668}{%
\subsubsection*{第四步:初始化优化器}\label{ux7b2cux56dbux6b65ux521dux59cbux5316ux4f18ux5316ux5668}}

    \begin{tcolorbox}[breakable, size=fbox, boxrule=1pt, pad at break*=1mm,colback=cellbackground, colframe=cellborder]
\prompt{In}{incolor}{6}{\boxspacing}
\begin{Verbatim}[commandchars=\\\{\}]
\PY{n}{learning\PYZus{}rate} \PY{o}{=} \PY{l+m+mf}{0.01}
\PY{n}{optimizer} \PY{o}{=} \PY{n}{nn}\PY{o}{.}\PY{n}{Adam}\PY{p}{(}\PY{n}{nn}\PY{o}{.}\PY{n}{default\PYZus{}graph}\PY{p}{,} \PY{n}{loss}\PY{p}{,} \PY{n}{learning\PYZus{}rate}\PY{o}{=}\PY{n}{learning\PYZus{}rate}\PY{p}{)}
\end{Verbatim}
\end{tcolorbox}

    \hypertarget{ux7b2cux4e94ux6b65ux5f00ux59cbux8badux7ec3}{%
\subsubsection*{第五步:开始训练}\label{ux7b2cux4e94ux6b65ux5f00ux59cbux8badux7ec3}}

    \begin{tcolorbox}[breakable, size=fbox, boxrule=1pt, pad at break*=1mm,colback=cellbackground, colframe=cellborder]
\prompt{In}{incolor}{7}{\boxspacing}
\begin{Verbatim}[commandchars=\\\{\}]
\PY{n}{batch\PYZus{}size} \PY{o}{=} \PY{l+m+mi}{2} \PY{c+c1}{\PYZsh{} 因为只拿12个句子,这里batch\PYZus{}size如果取16,模型就不更新了}
\PY{n}{epochs} \PY{o}{=} \PY{l+m+mi}{3}
\PY{n+nb}{print}\PY{p}{(}\PY{l+s+s2}{\PYZdq{}}\PY{l+s+s2}{更新前的随机词典:}\PY{l+s+s2}{\PYZdq{}}\PY{p}{)}
\PY{n+nb}{print}\PY{p}{(}\PY{n}{embedding}\PY{o}{.}\PY{n}{vocab}\PY{o}{.}\PY{n}{value}\PY{p}{)}
\PY{k}{for} \PY{n}{epoch} \PY{o+ow}{in} \PY{n+nb}{range}\PY{p}{(}\PY{n}{epochs}\PY{p}{)}\PY{p}{:}
    \PY{n}{count} \PY{o}{=} \PY{l+m+mi}{0}
    \PY{n}{N}\PY{o}{=} \PY{n+nb}{len}\PY{p}{(}\PY{n}{seqs}\PY{p}{)}
    \PY{n}{N} \PY{o}{=} \PY{l+m+mi}{12} \PY{c+c1}{\PYZsh{} 就拿前10条句子跑吧,否则还是跑不动}

    \PY{c+c1}{\PYZsh{} 填坑并训练}
    \PY{k}{for} \PY{n}{i} \PY{o+ow}{in} \PY{n+nb}{range}\PY{p}{(}\PY{n}{N}\PY{p}{)}\PY{p}{:}
        \PY{c+c1}{\PYZsh{} 句子的列表对象填坑}
        \PY{k}{for} \PY{n}{j} \PY{o+ow}{in} \PY{n+nb}{range}\PY{p}{(}\PY{n}{max\PYZus{}seq\PYZus{}len}\PY{p}{)}\PY{p}{:}
            \PY{k}{if} \PY{n}{seqs}\PY{p}{[}\PY{n}{i}\PY{p}{]}\PY{p}{[}\PY{n}{j}\PY{p}{]} \PY{o}{\PYZlt{}} \PY{n}{vocab\PYZus{}lens} \PY{p}{:}
                \PY{n}{seq}\PY{p}{[}\PY{n}{j}\PY{p}{]}\PY{o}{=}\PY{n}{seqs}\PY{p}{[}\PY{n}{i}\PY{p}{]}\PY{p}{[}\PY{n}{j}\PY{p}{]}
            \PY{k}{else}\PY{p}{:}
                \PY{n}{seq}\PY{p}{[}\PY{n}{j}\PY{p}{]}\PY{o}{=}\PY{l+m+mi}{0}
        \PY{c+c1}{\PYZsh{} 输入隐藏状态}
        \PY{n}{h\PYZus{}0}\PY{o}{.}\PY{n}{set\PYZus{}value}\PY{p}{(}\PY{n}{cp}\PY{o}{.}\PY{n}{zeros}\PY{p}{(}\PY{p}{(}\PY{l+m+mi}{1}\PY{p}{,} \PY{l+m+mi}{4}\PY{p}{)}\PY{p}{)}\PY{p}{)}
        \PY{n}{c\PYZus{}0}\PY{o}{.}\PY{n}{set\PYZus{}value}\PY{p}{(}\PY{n}{cp}\PY{o}{.}\PY{n}{zeros}\PY{p}{(}\PY{p}{(}\PY{l+m+mi}{1}\PY{p}{,} \PY{l+m+mi}{4}\PY{p}{)}\PY{p}{)}\PY{p}{)}
        \PY{n}{h\PYZus{}1}\PY{o}{.}\PY{n}{set\PYZus{}value}\PY{p}{(}\PY{n}{cp}\PY{o}{.}\PY{n}{zeros}\PY{p}{(}\PY{p}{(}\PY{l+m+mi}{1}\PY{p}{,} \PY{l+m+mi}{4}\PY{p}{)}\PY{p}{)}\PY{p}{)}
        \PY{n}{c\PYZus{}1}\PY{o}{.}\PY{n}{set\PYZus{}value}\PY{p}{(}\PY{n}{cp}\PY{o}{.}\PY{n}{zeros}\PY{p}{(}\PY{p}{(}\PY{l+m+mi}{1}\PY{p}{,} \PY{l+m+mi}{4}\PY{p}{)}\PY{p}{)}\PY{p}{)}
        \PY{c+c1}{\PYZsh{} 输入标签}
        \PY{n}{label}\PY{o}{.}\PY{n}{set\PYZus{}value}\PY{p}{(}\PY{n}{labels}\PY{p}{[}\PY{n}{i}\PY{p}{]}\PY{p}{)}
        \PY{c+c1}{\PYZsh{} 前向反向传播}
        \PY{n}{optimizer}\PY{o}{.}\PY{n}{one\PYZus{}step}\PY{p}{(}\PY{p}{)} 
        \PY{c+c1}{\PYZsh{} 更新计数器}
        \PY{n}{count} \PY{o}{+}\PY{o}{=} \PY{l+m+mi}{1}
        \PY{c+c1}{\PYZsh{} 计数器达到batch\PYZus{}size就更新模型参数}
        \PY{k}{if} \PY{n}{count} \PY{o}{\PYZgt{}}\PY{o}{=} \PY{n}{batch\PYZus{}size}\PY{p}{:} 
            \PY{n}{optimizer}\PY{o}{.}\PY{n}{update}\PY{p}{(}\PY{p}{)} 
            \PY{n}{count} \PY{o}{=} \PY{l+m+mi}{0}

    \PY{c+c1}{\PYZsh{} 每个epoch后评估模型的平均平方损失}
    \PY{n}{acc\PYZus{}loss} \PY{o}{=} \PY{l+m+mi}{0}
    \PY{k}{for} \PY{n}{i} \PY{o+ow}{in} \PY{n+nb}{range}\PY{p}{(}\PY{n}{N}\PY{p}{)}\PY{p}{:}
        \PY{k}{for} \PY{n}{j} \PY{o+ow}{in} \PY{n+nb}{range}\PY{p}{(}\PY{n}{max\PYZus{}seq\PYZus{}len}\PY{p}{)}\PY{p}{:}
            \PY{k}{if} \PY{n}{seqs}\PY{p}{[}\PY{n}{i}\PY{p}{]}\PY{p}{[}\PY{n}{j}\PY{p}{]} \PY{o}{\PYZlt{}} \PY{n}{vocab\PYZus{}lens} \PY{p}{:}
                \PY{n}{seq}\PY{p}{[}\PY{n}{j}\PY{p}{]}\PY{o}{=}\PY{n}{seqs}\PY{p}{[}\PY{n}{i}\PY{p}{]}\PY{p}{[}\PY{n}{j}\PY{p}{]}
            \PY{k}{else}\PY{p}{:}
                \PY{n}{seq}\PY{p}{[}\PY{n}{j}\PY{p}{]}\PY{o}{=}\PY{l+m+mi}{0}
        \PY{n}{h\PYZus{}0}\PY{o}{.}\PY{n}{set\PYZus{}value}\PY{p}{(}\PY{n}{cp}\PY{o}{.}\PY{n}{zeros}\PY{p}{(}\PY{p}{(}\PY{l+m+mi}{1}\PY{p}{,} \PY{l+m+mi}{4}\PY{p}{)}\PY{p}{)}\PY{p}{)}
        \PY{n}{c\PYZus{}0}\PY{o}{.}\PY{n}{set\PYZus{}value}\PY{p}{(}\PY{n}{cp}\PY{o}{.}\PY{n}{zeros}\PY{p}{(}\PY{p}{(}\PY{l+m+mi}{1}\PY{p}{,} \PY{l+m+mi}{4}\PY{p}{)}\PY{p}{)}\PY{p}{)}
        \PY{n}{h\PYZus{}1}\PY{o}{.}\PY{n}{set\PYZus{}value}\PY{p}{(}\PY{n}{cp}\PY{o}{.}\PY{n}{zeros}\PY{p}{(}\PY{p}{(}\PY{l+m+mi}{1}\PY{p}{,} \PY{l+m+mi}{4}\PY{p}{)}\PY{p}{)}\PY{p}{)}
        \PY{n}{c\PYZus{}1}\PY{o}{.}\PY{n}{set\PYZus{}value}\PY{p}{(}\PY{n}{cp}\PY{o}{.}\PY{n}{zeros}\PY{p}{(}\PY{p}{(}\PY{l+m+mi}{1}\PY{p}{,} \PY{l+m+mi}{4}\PY{p}{)}\PY{p}{)}\PY{p}{)}
        \PY{n}{label}\PY{o}{.}\PY{n}{set\PYZus{}value}\PY{p}{(}\PY{n}{labels}\PY{p}{[}\PY{n}{i}\PY{p}{]}\PY{p}{)}
        \PY{n}{loss}\PY{o}{.}\PY{n}{forward}\PY{p}{(}\PY{p}{)}
        \PY{n}{acc\PYZus{}loss} \PY{o}{+}\PY{o}{=} \PY{n}{loss}\PY{o}{.}\PY{n}{value}
    \PY{n}{average\PYZus{}loss} \PY{o}{=} \PY{n}{acc\PYZus{}loss} \PY{o}{/} \PY{n}{N}
    \PY{n+nb}{print}\PY{p}{(}\PY{l+s+s2}{\PYZdq{}}\PY{l+s+s2}{epoch:}\PY{l+s+si}{\PYZob{}\PYZcb{}}\PY{l+s+s2}{ , average\PYZus{}loss:}\PY{l+s+si}{\PYZob{}\PYZcb{}}\PY{l+s+s2}{\PYZdq{}}\PY{o}{.}\PY{n}{format}\PY{p}{(}\PY{n}{epoch}\PY{o}{+}\PY{l+m+mi}{1}\PY{p}{,} \PY{n}{cp}\PY{o}{.}\PY{n}{sqrt}\PY{p}{(}\PY{n}{average\PYZus{}loss}\PY{p}{)}\PY{p}{[}\PY{l+m+mi}{0}\PY{p}{]}\PY{p}{[}\PY{l+m+mi}{0}\PY{p}{]}\PY{p}{)}\PY{p}{)}
\PY{n+nb}{print}\PY{p}{(}\PY{l+s+s2}{\PYZdq{}}\PY{l+s+s2}{更新后的词典:}\PY{l+s+s2}{\PYZdq{}}\PY{p}{)}
\PY{n+nb}{print}\PY{p}{(}\PY{n}{embedding}\PY{o}{.}\PY{n}{vocab}\PY{o}{.}\PY{n}{value}\PY{p}{)}
\end{Verbatim}
\end{tcolorbox}

    \begin{Verbatim}[commandchars=\\\{\}]
更新前的随机词典:
[[ 0.          0.          0.          0.          0.          0.        ]
 [ 0.04601878  0.00404048  0.04015696 -0.06519445 -0.03926611  0.0139227 ]
 [-0.06951123  0.08241394  0.02143262  0.06112293  0.01719459 -0.03234663]
 [-0.00495689 -0.00952639 -0.08017523 -0.0049339   0.06948158 -0.01158497]
 [-0.05801717 -0.04864316 -0.01222767  0.09520323 -0.0561946   0.0125205 ]
 [-0.09940845  0.03058567 -0.06553359 -0.09607915 -0.03470116 -0.07746006]
 [ 0.02574757  0.08103257 -0.00861659 -0.08763861  0.00053078  0.06172405]
 [-0.01261335 -0.01195699 -0.00790909  0.06745773  0.01438359 -0.0432668 ]
 [-0.07392157  0.00809994  0.08484132  0.08505075 -0.01008637  0.06179531]
 [ 0.0030789   0.0487923   0.02765147 -0.02074502  0.0231011   0.09824966]
 [-0.03578965 -0.08761168  0.02594208 -0.01307079  0.06163082 -0.01358586]
 [ 0.05578862  0.02245893  0.092424   -0.05362544 -0.00972471 -0.04428356]
 [ 0.07484926 -0.02366823 -0.09911224  0.04430081 -0.00841923 -0.00920663]
 [ 0.02676214  0.06327197  0.08703353  0.06904646 -0.03202954  0.03444296]
 [-0.01285214  0.07879407  0.04201142  0.00959265  0.06963184  0.01009394]
 [-0.08374249 -0.05770571 -0.02031106 -0.01629067  0.05843795 -0.07895623]
 [-0.04759348 -0.08748662  0.00497911  0.00683976 -0.0933935  -0.0852092 ]
 [-0.02791182 -0.06868168 -0.05379565 -0.00373487 -0.04043877  0.0174353 ]
 [-0.08165128 -0.03259038 -0.09113629 -0.08790202  0.00799482 -0.04439018]
 [ 0.0215529   0.01567511 -0.06697443  0.02731935 -0.03042926  0.07366363]]
epoch:1 , average\_loss:0.9893749000353604
epoch:2 , average\_loss:0.8663498296578028
epoch:3 , average\_loss:0.8411084832551297
更新后的词典:
[[-0.05649016 -0.04927701 -0.04806061  0.15907731  0.00178527  0.05822795]
 [-0.00156807 -0.06583477  0.15503936  0.04629727 -0.00317504  0.07930903]
 [-0.14579108  0.02817459  0.06145587  0.22798977  0.09353109  0.02435572]
 [-0.01749295 -0.10734621 -0.14469249 -0.04212084  0.12125494  0.05556165]
 [-0.01051013 -0.11818024 -0.0661573  -0.00720834  0.01919932  0.04678992]
 [-0.06861676 -0.06855189 -0.10726222 -0.27821925  0.01145983 -0.01735449]
 [-0.1242651   0.13774998 -0.00981059 -0.02559029  0.03498846  0.07474569]
 [-0.07225565 -0.00482027  0.07053885  0.14182547  0.08221801  0.02598057]
 [-0.07392157  0.00809994  0.08484132  0.08505075 -0.01008637  0.06179531]
 [ 0.0030789   0.0487923   0.02765147 -0.02074502  0.0231011   0.09824966]
 [-0.03578965 -0.08761168  0.02594208 -0.01307079  0.06163082 -0.01358586]
 [ 0.05578862  0.02245893  0.092424   -0.05362544 -0.00972471 -0.04428356]
 [ 0.07484926 -0.02366823 -0.09911224  0.04430081 -0.00841923 -0.00920663]
 [ 0.02676214  0.06327197  0.08703353  0.06904646 -0.03202954  0.03444296]
 [-0.01285214  0.07879407  0.04201142  0.00959265  0.06963184  0.01009394]
 [-0.08374249 -0.05770571 -0.02031106 -0.01629067  0.05843795 -0.07895623]
 [-0.04759348 -0.08748662  0.00497911  0.00683976 -0.0933935  -0.0852092 ]
 [-0.02791182 -0.06868168 -0.05379565 -0.00373487 -0.04043877  0.0174353 ]
 [-0.08165128 -0.03259038 -0.09113629 -0.08790202  0.00799482 -0.04439018]
 [ 0.0215529   0.01567511 -0.06697443  0.02731935 -0.03042926  0.07366363]]
    \end{Verbatim}


    % Add a bibliography block to the postdoc
    
    
    
\end{document}
